\chapter{Tokenizing}
What is a lexer? The job of a lexer is to read you code and identify what we call tokens. Tokens are simply named pieces of text. We'll be using a program called Flex to tokenize our code. Flex is a lexical analyzer designed to create a tokenizer.

A Flex file will typically have this structure:

\begin{lstlisting}
definitions and directives
%%
rules
%%
user code
\end{lstlisting}

As a first step we'll want to create a file called {\bf ruby.l}. Add the following content:

\begin{lstlisting}[language=C]
%{
  #include <stdlib.h>
  #include <stdio.h>
%}

%option noyywrap

%%
[0-9]+ { printf("NUMBER: %s\n", yytext); }
[ \t\n] {}
. {}
%%

int main(int argc, char *argv[]) {
  yylex();
  return EXIT_SUCCESS;
}
\end{lstlisting}

Now to compile and run this do the following

\begin{lstlisting}
flex ruby.l
gcc -o ruby lex.yy.c
./ruby
\end{lstlisting}

Then try the following:

\begin{lstlisting}
1
NUMBER: 1
42
NUMBER: 42
\end{lstlisting}

Then press ctrl-c to exit
