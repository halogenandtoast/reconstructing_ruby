\documentclass[oneside]{book}

\usepackage[dvipsnames]{xcolor}
\usepackage{listings}
\lstloadlanguages{C}
\lstset{%
showspaces=false,
showstringspaces=false,
basicstyle=\ttfamily\color{black},
commentstyle=\ttfamily\color{red},
keywordstyle=\ttfamily\color{blue},
stringstyle=\ttfamily\color{orange}}


\date{}
\begin{document}
\title{Rebuilding Ruby}
\author{Matthew Mongeau}
\maketitle

\frontmatter
\tableofcontents
\chapter*{Preface}
\addcontentsline{toc}{chapter}{Preface}

This book is not a primer on C. It assumes that you have a passing knowledge of C programming. If you've never programmed in C before it's highly suggested that you read the book "C Programming Language, 2nd Edition" by Brian W. Kernighan and Dennis M. Ritchie -  which is often referred to as "K\&R".

This book has been written for two purposes. The first is to provide a better understanding of Ruby's internals, potentially allowing readers to contribute to Ruby. A second goal is to provide a better understanding of interpreters in general. The second goal opens up the possibilities for readers to create their own interpreted languages.

\mainmatter
\chapter{Tokenizing}
What is a lexer? The job of a lexer is to read you code and identify what we call tokens. Tokens are simply named pieces of text.

\newpage
This is on another page.

\begin{lstlisting}[language=C]
#include <stdio.h>

int main(int argc, char *argv[]) {
  printf("Hello World\n");
  return 0;
}
\end{lstlisting}

\chapter{Parsing}
What is parsing? Parsing is verifying the structure of our code.

\appendix
\chapter{References}

\backmatter
\chapter{Notes}
\end{document}
